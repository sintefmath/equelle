% Listing package
\usepackage{listings}
\lstdefinestyle{small}{
  language=python, % choose the language of the code
  basicstyle=\scriptsize, % the size of the fonts that are used for the code
  numbers=left, % where to put the line-numbers
  % numberstyle=\footnotesize, % the size of the fonts that are used for the line-numbers
  stepnumber=1, % the step between two line-numbers. If it's 1 each line will be numbered
  numbersep=5pt, % how far the line-numbers are from the code
  % backgroundcolor=\color{white}, % choose the background color. You must add \usepackage{color}
  morekeywords={*, Function,SecondCell, FirstCell, Collection, Of, Scalar, On, AllCells, BoundaryFaces, InteriorFaces, InputCollectionOfScalar, Output, InputScalarWithDefault, InputSequenceOfScalar, AllFaces, Sequence, Array, Extend, For, In, Normal, Centroid, Sqrt, IsEmpty, NewtonSolve, NewtonSolveArray, Divergence, Gradient, Mutable, Dot, InputDomainSubsetOf, Face, Subset},
  keywordstyle=\ttfamily\color{blue},
  showspaces=false, % show spaces adding particular underscores
  showstringspaces=false, % underline spaces within strings
  showtabs=false, % show tabs within strings adding particular underscores
  frame=single, % adds a frame around the code
  % frame=L,
  xleftmargin=\parindent,
  % tabsize=2, % sets default tabsize to 2 spaces
  % captionpos=b, % sets the caption-position to bottom
  breaklines=true, % sets automatic line breaking
  breakatwhitespace=true, % sets if automatic breaks should only happen at whitespace
  escapeinside={\%*}{*)} % if you want to add a comment within your code
}
